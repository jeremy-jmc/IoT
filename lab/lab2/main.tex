\documentclass{article}
\usepackage{amssymb} % Required for math symbols
\usepackage{graphicx} % Required for inserting images
\usepackage[hidelinks]{hyperref}
\usepackage{float}

\usepackage[utf8]{inputenc}
\usepackage{amsmath}
\usepackage[a4paper, total={6in, 10in}]{geometry}
\usepackage{listings}
\usepackage{xcolor}

\definecolor{codegray}{rgb}{0.5,0.5,0.5}
\definecolor{codepurple}{rgb}{0.58,0,0.82}
\definecolor{backcolour}{rgb}{0.95,0.95,0.92}

\lstdefinestyle{cppstyle}{
    backgroundcolor=\color{backcolour},   
    commentstyle=\color{codegray}\ttfamily,
    keywordstyle=\color{blue}\bfseries,
    numberstyle=\tiny\color{gray},
    stringstyle=\color{codepurple},
    basicstyle=\ttfamily\footnotesize,
    breaklines=true,
    captionpos=b,
    keepspaces=true,
    numbers=left,
    numbersep=5pt,
    showspaces=false,
    showstringspaces=false,
    showtabs=false,
    tabsize=2,
    language=C++
}

\title{Laboratorio 2 - Sensores y Actuadores}
\author{jeremy.matos@utec.edu.pe, luis.gutierrez@utec.edu.pe}
\date{Abril 2025}

\begin{document}

\maketitle

\newpage
\tableofcontents
\newpage

\section{Introducción}

\subsection{Objetivo General}

\subsection{Objetivos Específicos}

\newpage

\section{Marco teórico}

\section{Estado del Arte}

% file:///home/tenken/Downloads/30_ShodhKosh_RS_5111.pdf

% https://www.etasr.com/index.php/ETASR/article/view/6410

% https://users.cecs.anu.edu.au/~sid.chau/papers/TOSN-smlight.pdf

% https://www.semanticscholar.org/paper/Innovative-IoT-Smart-Lock-System:-Enhancing-with-Zainuddin-Rahman/563edb5ed5795c9bb35eefc69aa010c9f7c60157

\section{Metodología}

\subsection{SENSORES - Sensor de luz ambiental}

% DONE: Simulacion

En esta experiencia de laboratorio, se desarrollará un sistema de medición ambiental utilizando los siguientes componentes: 
\begin{itemize}
    \item 1 LDR (sensor de luz)
    \item Diodos LED
    \item Buzzer (alarma)
    \item Botón o switch
\end{itemize}

El objetivo es implementar un sistema que permita medir la cantidad de luz en un ambiente determinado, el cual complementará un sistema de cámaras de seguridad y un control inteligente de luminarias. El sistema debe cumplir con las siguientes funciones:
\begin{enumerate}
    \item Si el nivel de iluminación es bajo, las luminarias de la habitación deben encenderse, representado por el encendido de un LED rojo.
    \item Si el nivel de iluminación es alto, las luminarias deben permanecer apagadas, representado por el LED rojo apagado.
    \item Las cámaras de seguridad deben estar en funcionamiento en todo momento, lo cual se representa mediante el encendido de un LED azul.
    \item En caso de que no se detecte energía eléctrica para el sistema de cámaras (entrada digital detecta GND), el sistema debe encender y apagar el LED azul de manera intermitente, además de activar un buzzer que representa una alarma de error. Esto puede simularse utilizando un botón, un switch o una desconexión de cable.
\end{enumerate}

La implementación se realizó en la plataforma Tinkercad, utilizando un Arduino UNO. La conexión de los componentes se puede observar en la Figura \ref{fig:luz_ambiental}. El sensor LDR se conectó a una entrada analógica (A0), mientras que los LEDs y el buzzer se conectaron a salidas digitales. Un botón se empleó como entrada digital para simular la presencia o ausencia de energía del sistema de cámaras.

\begin{figure}[H]
    \centering
    \includegraphics[width=0.85\textwidth]{./img/ckpt_1.png}
    \caption{Simulación Tinkercad Sensor de luz ambiental}
    \label{fig:luz_ambiental}
\end{figure}

En el código (Código \ref{code:sensor_ambiental}), se define un umbral de luminosidad para distinguir entre un ambiente claro u oscuro. Si el valor leído del LDR es menor que este umbral, se activa el LED rojo. Por otro lado, si el botón indica que hay energía (estado HIGH), el LED azul permanece encendido de forma constante. En caso contrario (estado LOW), el sistema simula una falla eléctrica encendiendo el buzzer y haciendo parpadear el LED azul.

\begin{lstlisting}[style=cppstyle, caption={Código en C++ para el sensor ambiental.}, label={code:sensor_ambiental}]
int LDR_pin = A0;
int ROJO_pin = 13;
int AZUL_pin = 12;
int BUZZER_pin = 11;
int BUTTON_pin = 7;
int luz, estadoBoton;
const int umbral = 500;  // Ajustarlo

void setup() {
    pinMode(ROJO_pin, OUTPUT); pinMode(AZUL_pin, OUTPUT);
    pinMode(BUZZER_pin, OUTPUT);
    pinMode(BUTTON_pin, INPUT_PULLUP);
    Serial.begin(9600);
}

void loop() {
    luz = analogRead(LDR_pin);
    estadoBoton = digitalRead(BUTTON_pin);
    Serial.println("Luz: " + String(luz) + " - " + String(umbral));
    
    if (luz < umbral) digitalWrite(ROJO_pin, HIGH);
    else digitalWrite(ROJO_pin, LOW);
    
    if (estadoBoton == HIGH) {
        digitalWrite(AZUL_pin, HIGH);
        digitalWrite(BUZZER_pin, LOW);
    } else {
        digitalWrite(BUZZER_pin, HIGH);
        digitalWrite(AZUL_pin, HIGH); delay(300);
        digitalWrite(AZUL_pin, LOW); delay(300);
    }
    delay(1000);
}    
\end{lstlisting}

% TODO: foto de la implementacion presencial en el laboratorio

\subsection{SENSORES - Cerradura inteligente}

% TODO: Diagrama de flujo
% DONE: Simulacion
En esta experiencia de laboratorio, se desarrollará un sistema de seguridad para el control de acceso de personas a un área restringida. Para ello, se utilizarán los siguientes componentes: 

\begin{itemize}
    \item 01 sensor de proximidad.
    \item 01 teclado matricial.
    \item 01 LCD 2x16.
    \item Diodos LED.
    \item Resistencias variadas.
    \item Jumpers.
\end{itemize}

La solución final se puede ver en la Figura \ref{fig:cerradura_smart}. En las siguientes subsecciones se desglosarán las diferentes funcionalidades del sistema.

\begin{figure}[H]
    \centering
    \includegraphics[width=0.85\textwidth]{./img/ckpt_6.png}
    \caption{Simulación Tinkercad Resultado final del sistema de cerradura inteligente}
    \label{fig:cerradura_smart}
\end{figure}

% TODO: foto de la implementacion presencial en el laboratorio

\subsubsection{Presencia de la persona}

\subsubsection{Mensaje en LCD}

\subsubsection{Autenticación}

\subsubsection{Ahorro Energético (Standby)}

\subsection{SENSORES - Sistema de control de aforo}

% DONE: Simulacion

\begin{figure}[H]
    \centering
    \includegraphics[width=0.85\textwidth]{./img/ckpt_9.png}
    \caption{Simulación Tinkercad Resultado final del sistema de control de aforo}
    \label{fig:control_aforo}
\end{figure}

\begin{lstlisting}[style=cppstyle, caption={Código en C++ para el sistema de control de aforo.}, label={code:sensor_ambiental}]
#include <LiquidCrystal.h>

// Configuracion del LCD
LiquidCrystal lcd(8, 9, 4, 5, 6, 7);
const int PIRPin = 2;
const int LEDGreen = 13;
const int LEDRed = 12;

// Variable to store the number of incidents
int incidentCount = 0;

void setup() {
    Serial.begin(9600);
    
    lcd.begin(16, 2);
    lcd.clear();
    pinMode(PIRPin, INPUT);
    pinMode(LEDGreen, OUTPUT);
    pinMode(LEDRed, OUTPUT);
}

void loop() {
    int value = digitalRead(PIRPin);
    Serial.println(value);

    if (value == LOW) {
    // No movement detected, indicating the rest mode
    digitalWrite(LEDGreen, HIGH);   // Green LED blink (indicates idle state)
    delay(200);
    digitalWrite(LEDGreen, LOW);
    delay(200);
    
    // Display "RESTING" on LCD when no movement detected
    lcd.clear();
    lcd.setCursor(0, 0);
    lcd.print("Modo de Reposo");
    } else {
    // Movement detected, indicating the active state
    digitalWrite(LEDGreen, LOW);    // Turn off the green LED
    digitalWrite(LEDRed, HIGH);     // Turn on the red LED

    // Increment incident count when movement is detected
    incidentCount++;

    // Display incident count on LCD
    lcd.clear();
    lcd.setCursor(0, 0);
    lcd.print("Incidencias: ");
    lcd.setCursor(0, 1);
    lcd.print("00" + String(incidentCount));  // Assuming a 3-digit max count
    
    delay(2000); // Wait 2 seconds before checking again
    }
    
    delay(10);  // Small delay to stabilize the loop
}    
\end{lstlisting}
    
% TODO: foto de la implementacion presencial en el laboratorio
% TODO: diagrama de flujo
% TODO: diagrama de bloques
% ? TODO: checkpoint 5 y 6 son distintas simulaciones????


\subsection{ACTUADORES - Encendido de un Motor DC con Arduino}

% https://www.tinkercad.com/things/7Ta4Am5NLdF/editel?returnTo=%2Fdashboard%2Fdesigns%2Fcircuits&sharecode=ddM9-NkLnJGgaDXANovwieeo9xofWe2gha9ahUL3Y-o
\begin{figure}[H]
    \centering
    \includegraphics[width=0.85\textwidth]{./img/ckpt_encendido_motor.png}
    \caption{Simulación Tinkercad ...}
    \label{fig:encendido_motor}
\end{figure}

% TODO: simulacion
% TODO: diagrama de bloques
% TODO: respuesta a las preguntas

\begin{enumerate}
    \item ¿Cuál es la corriente que consume el motor durante su operación?
    \item Si usted quisiera regular la velocidad del motor, ¿qué añadiría a su esquema de conexión?
\end{enumerate}

\subsection{ACTUADORES - Control de un motor con Driver}

% https://www.tinkercad.com/things/6vFVydoth82/editel?returnTo=%2Fdashboard%2Fdesigns%2Fcircuits&sharecode=zm5Em3SqUbJ5xXMONP0ul2aEuINO3AnNjSSn-egeFUc

\begin{figure}[H]
    \centering
    \includegraphics[width=0.85\textwidth]{./img/ckpt_motor_driver.png}
    \caption{Simulación Tinkercad ...}
    \label{fig:motor_driver}
\end{figure}

\begin{enumerate}
    \item ¿Cuál es la función del potenciómetro en este sistema de control de motor y cómo se
    relaciona con la velocidad de rotación?
    \item Explique por qué se requiere una fuente de alimentación externa para el motor en lugar de usar directamente la salida de 5V del Arduino.
    \item Describa el comportamiento esperado del motor cuando se presiona únicamente el botón
    derecho, el izquierdo, y cuando no se presiona ninguno. ¿Qué sucede si se presionan los dos?
\end{enumerate}

% TODO: simulacion
% TODO: diagrama de bloques
% TODO: respuesta a las preguntas


\subsection{ACTUADORES - Uso de Relé con Motor}

\begin{figure}[H]
    \centering
    \includegraphics[width=0.85\textwidth]{./img/ckpt_rele_motor.png}
    \caption{Simulación Tinkercad ...}
    \label{fig:motor_driver}
\end{figure}

% TODO: simulacion
% TODO: foto de la implementacion presencial en el laboratorio
% TODO: diagrama de flujo
% TODO: diagrama de bloques

\subsection{ACTUADORES - Uso de Driver Puente H L298N con Motor}

\begin{enumerate}
    \item Prueba alimentar el módulo L298N directamente desde el pin de 5V del Arduino (en lugar de la fuente para protoboard). ¿Notas alguna diferencia en el rendimiento del motor? Justifica técnicamente tu observación.
    \item Restablezca la alimentación del motor. Modifica el circuito para invertir el sentido de giro automáticamente cada 3 segundos.
    \item ¿Qué función cumple el pin ENA en el módulo L298N y por qué debe conectarse a un pin PWM del Arduino?
    \item ¿Qué sucede si se intercambian las conexiones de IN1 e IN2? ¿Cómo afecta esto a la dirección del giro del motor?
\end{enumerate}

% TODO: implementacion
% TODO: respuesta a las preguntas
% TODO: diagrama de flujo
% TODO: diagrama de bloques


\subsection{ACTUADORES - Sistema de riego inteligente}

% TODO: foto de la implementacion presencial en el laboratorio
% TODO: diagrama de flujo
% TODO: diagrama de bloques


\subsection{ACTUADORES - Control de ventiladores}

% TODO: solo simulacion Tinkercad
% TODO: diagrama de bloques del sistema


\section{Conclusiones}

\end{document}